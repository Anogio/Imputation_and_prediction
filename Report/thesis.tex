\documentclass[12pt, a4paper]{article}

\usepackage{microtype}
\usepackage{amsmath,amsthm,amssymb,amsfonts}
\usepackage{bbm}
\usepackage{hyperref}
\usepackage{graphicx, subfig}
\usepackage{float}
\usepackage{listings}
\usepackage{minted}
\usepackage{xcolor}
\usepackage{todonotes}

\title{Missing data imputation for prediction: \\ The case of haemorrhagic shock prediction}
\author{Antoine Ogier \\ Supervisor: Julie Josse \\ Academic supervisor: Geoff Nicholls}

\begin{document}
\maketitle
\thispagestyle{empty}
\tableofcontents
\newpage
\pagenumbering{arabic}

\section*{Acknowledgements}
\section*{Introduction}
Missing values in data is a prominent issue that has been much discussed in statistical literature. In the eighties, Donald Rubin devised many of the tools that are still used today to handle missing data: the expectation-maximisation algorithm, the definition of the three missing data patterns (MCAR, MAR, MNAR), multiple imputation. 

However, there is in this regard a significant gap between the fields of statistical inference and machine learning: while the former has been actively developing and evaluating methods to handle missing data --- mostly for parameter and confidence intervals estimation ---, these methods are rarely used in the context of prediction, where imputing missing values with the mean of the observed data is standard practice.


\section{The data}

\end{document}