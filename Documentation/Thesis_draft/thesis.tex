\documentclass[12pt, a4paper]{memoir}

\setcounter{secnumdepth}{2}
\usepackage{natbib}
\setcitestyle{numbers,open={[},close={]}}
\bibliographystyle{unsrtnat}

\usepackage[utf8]{inputenc}
\usepackage[T1]{fontenc}
\usepackage[english]{babel}
\usepackage{lmodern}
%\usepackage[marginparwidth=70pt]{geometry}

\usepackage{microtype}
\usepackage{amsmath,amsthm,amssymb,amsfonts}

\usepackage{bbm}
\usepackage{atbegshi}
\usepackage{hyperref}
\usepackage{graphicx}
\usepackage{float}
\usepackage{listings}
\usepackage{minted}
\usepackage{xcolor}
\usepackage{todonotes}

\DeclareMathOperator*{\argmax}{arg\,max}
\DeclareMathOperator*{\argmin}{arg\,min}

\newsubfloat{figure} % used to enable subfloats in the memoir class
\begin{document}
\pagenumbering{roman}

\AtBeginShipoutNext{\AtBeginShipoutNext{\AtBeginShipoutDiscard}}
\begin{titlingpage}
	\newcommand{\HRule}{\rule{\linewidth}{0.5mm}} % Defines a new command for the horizontal lines, change thickness here
	
	\center % Center everything on the page
	 
	%----------------------------------------------------------------------------------------
	%	HEADING SECTIONS
	%----------------------------------------------------------------------------------------
	
	\textsc{\LARGE University of Oxford}\\[1.5cm] % Name of your university/college
	\textsc{\Large MSc in Statistical Science}\\[0.5cm] % Major heading such as course name
	\textsc{\large Final thesis}\\[0.5cm] % Minor heading such as course title
	
	%----------------------------------------------------------------------------------------
	%	TITLE SECTION
	%----------------------------------------------------------------------------------------
	
	\HRule \\[0.4cm]
	{ \huge \bfseries Missing data imputation for Haemorrhagic shock prediction}\\[0.4cm] % Title of your document
	\HRule \\[1.5cm]
	 
	%----------------------------------------------------------------------------------------
	%	AUTHOR SECTION
	%----------------------------------------------------------------------------------------
	
	\begin{minipage}{0.4\textwidth}
	\begin{flushleft} \large
	\emph{Author:}\\
	Antoine \textsc{Ogier} % Your name
	\end{flushleft}
	\end{minipage}
	~
	\begin{minipage}{0.4\textwidth}
	\begin{flushright} \large
	\emph{Supervisor:} \\
	Pr. Julie \textsc{Josse} \\(École polytechnique) \\% Supervisor's Name
	Pr. Geoff \textsc{Nicholls} \\(University of Oxford) % Supervisor's Name
	
	\end{flushright}
	\end{minipage}\\[1cm]
	
	% If you don't want a supervisor, uncomment the two lines below and remove the section above
	%\Large \emph{Author:}\\
	%John \textsc{Smith}\\[3cm] % Your name
	
	%----------------------------------------------------------------------------------------
	%	DATE SECTION
	%----------------------------------------------------------------------------------------
	
	{\large September 2018}\\[1cm] % Date, change the \today to a set date if you want to be precise
	
	%----------------------------------------------------------------------------------------
	%	LOGO SECTION
	%----------------------------------------------------------------------------------------
	
	\includegraphics[scale=1.2]{logo}\\[1cm] % Include a department/university logo - this will require the graphicx package
	%----------------------------------------------------------------------------------------
	%\vfill % Fill the rest of the page with whitespace
\end{titlingpage}


\begin{vplace}[0.7]
\begin{abstract}
Lorem ipsum dolot sit amet nunc cui Brexit.
\end{abstract}
\end{vplace}

\newpage
\vspace*{\fill}
{\centering\huge\bfseries Acknowledgements\par}
\bigskip
\noindent Lorem ipsum dolot sit amet
\vspace*{\fill}
\newpage

%\vspace*{\fill}
\tableofcontents*
\vspace*{\fill}

\addtocontents{toc}{\protect\vspace*{\fill}}
\chapter*{Introduction}
\pagenumbering{arabic}
\addcontentsline{toc}{chapter}{Introduction}

\chapter{Goal and data}
\label{data}
	\section{The problem of haemorrhagic shock}
	Prediction is very hard, as described in \cite{doctors_prediction}
	\section{The Traumabase data}
	\section{Exploratory data analysis}
		\subsection{Variables}
		\subsection{Missing data}
	\section{Our goal: imputation}
%	\textbf{Note:} tout le but de l'imputation est de permettre l'utilisation de n'importe quel modèle  prédictif une fois que $X$ est imputé. Une manière de voir la situation est la suivante: deux personnes travaillent sur les données de manière successive, l'Imputeur et l'Analyste. L'Imputeur ne sait pas quel modèle l'analyste va choisir d'utiliser, il va donc optimiser l'imputation en utilisant seulement les informations dont il dispose. Une fois les données imputées, elle sont transmise à l'Analyste qui procède à la prédiction. 
%réadapter paragraphe pour expliquer l'idée de pourquoi on fait de l'imputation plutôt que du joint modelling

\chapter{Imputation methods}
\label{imputation}
	\section{Main types of imputation}
		\subsection{Joint parametric specification}
		\subsection{Fully conditional specification: the MICE algorithm}
		\subsection{Low-rank approximation for imputation}
		\subsection{ML-based}
	\section{Multiple imputation}
		\subsection{Principle}
		\subsection{Rubin's rule and prediction aggregation}
	\section{Normality hypothesis: transforming the data}
		
\chapter{Methodology: imputation and the validation split}
\label{validation}
The task we are trying to solve is quite particular: we are trying to impute the missing values in the data, not to perform a statistical analysis, but to select and train a prediction model. Our final benchmark of performance is not parameter estimation but predictive loss. As the next two chapters will show, although this seems like a minor difference, this actually leads to some major changes.

It is interesting to note in that regard how the communities of statistics and machine learning seem to lack any point of convergence on the subject. Missing data have been an active field of research in statistics for a long time \cite{rubin1976inference}, and many complex methods have been developed and proven for statistical inference \cite{Rubin_missdata} (such as those described in Chapter \ref{imputation}).

On the other hand, there is almost no research on these methods when applied to prediction. Even recent manuals for machine learning \cite{ML_missdata} generally make only a quick mention of missing data, and in practice it is extremely rare for anything else than imputation by the mean to be used. \emph{Scikit-learn} \cite{scikit-learn}, by far the most-used machine learning package in the community, only proposes implementation by the mean as of now. 

A few research papers \cite{prediction_imputation1} \cite{prediction_imputation2} try to assess the performance of more modern imputation methods when used in a predictive context. However, they do not propose any framework or theory on this endeavour. They take it for granted that they can impute the whole dataset before performing the subsequent analysis. However, when performing prediction there is one significant difference with statistical inference, which we describe further in this chapter: the data is split into one dataset to learn the model, and another one to validate its performance. This raises many questions that we discuss here and in Chapter \ref{linreg}. 

After laying out the general framework of Empirical Risk Minimization \cite{ERM}, which is the general paradigm used for prediction, we adapt it to fit the context of missing data. We notice that current implementations of modern imputation methods are incompatible with this framework, and try to devise solutions for this issue.
	\section{Empirical risk minimization (ERM): classical context}
We first describe Empirical Risk Minimization (ERM) without missing data, as described in \cite{ERM}.
		\subsection{Context and notations}
We are provided with a matrix $X$ of size $n \times p$ and response vector $y$ of size $n$. Our goal is to learn a model in the form $\hat{y} = f(x, \psi)$, where $\psi$ is some parameter to choose, $\hat{y}$ is a predicted value for $y$ and $f$ is a fixed parametric predictive function (usually corresponding to a choice of function in a given class).

The quality of a prediction is evaluated with the loss $L(y,\hat{y})$. The end goal in this context is to find the parameter $\hat{\psi}$ which minimizes the risk: $R = \mathbb{E}(L(y, f(X)))$. However, we do not have access to the real expectation of the risk, so we must use a proxy for this value. We define the empirical risk:

$$ R_{\text{emp}}(y, f(X, \psi)) = \frac{1}{n} \sum\limits_{i=1}^n L(y_i, f(X_i, \psi))$$

Empirical Risk Minimization corresponds to choosing $\psi$ minimizing $R_{emp}$ for our $X$ and $y$. However, this is not enough. We do not just want the optimal $\psi$, we also want a measure of how well the final model would perform on new data. This is important because this is what we will take into account to make our choice of $f$. But the empirical risk gives us no measure of how well our model generalizes whatsoever, only of how closely it can fit known data.

To address the issue of model selection, the standard practice is to measure the error on some data that were not used to learn the model: it is the principle of cross-validation.

		\subsection{Cross-validation}
To perform Cross-validation, we divide the data in two datasets: first we choose $n_A < n$ entries in the dataset to be used to loearn $\psi$: this is the training dataset $X_A$ and response $y_A$. We denote $I_A = (i_1, \ldots, i_{n_A})$ the set of indices chosen for the training data. The rest of the observations are noted $X_V$ and $y_V$ and called the validation dataset.

Once this is done, ERM is performed as before, using only the training data. The obtained parameter $\hat{\psi}$ can then be evaluated with the validation error:

$$ R_{V} = \frac{1}{n_V} \sum\limits_{i=1, i \notin I_A}^n L(y_i, f(X_i, \hat{\psi}))$$

It is this value that we can compare to choose our our model class $f$.

	\section{ERM with missing data: the problem of current methodologies}
We now place ourselves in the same context as before, except some values are missing from $X$, both in the training and the validation data. The context is almost the same as before: choosing a parametric model that takes as input the observed data and outputs a prediction for $y$.
		\subsection{Imputation seen as an ERM}
Remember that the purpose of this work is to impute the data independently of the predictive model used afterwards. This does not change the framework of ERM but it does mean that we cannot use any function we like to go from $X$ to $\hat{y}$. The prediction is the composition of two steps.
			\paragraph{Imputation step}
First we choose an imputation model $X^{\text{complete}} = g(X, \phi)$ where $X^{\text{complete}}$ is the completed dataset and $\phi$ some parameter. This is similar to predicting $y$ as we did previously with one caveat: we do not know the true data, even on the training dataset. Thus we choose $\hat{\phi}$ to minimize some unsupervised loss
$$L'(g(X, \phi), \phi)$$
 measuring the fit of the model to the data. Generally, this means that we choose a parameter that maximizes the likelihood of the observed data according to some generative model (though it is not always the case). Once this is done, we obtain a completed dataset $\hat{X}$.

			\paragraph{Prediction step}
Once the imputation is done, we can perform as before to choose a parameter $\hat{\psi}$ that minimizes the empirical risk when using the completed data:
	$$ R_{\text{emp}}(y, f(\hat{X}_i, \psi)) = \frac{1}{n} \sum\limits_{i=1}^n L(y_i, f(\hat{X}_i, \psi))$$
	
Putting it all together, we can define 
$$ h(X, (\psi, \phi)) = f(X^{imp}, \psi) = f( g(X, \phi), \psi) $$

the combined model that takes the observed data as input and outputs a predicted $y$. This is optimized as 
\begin{align*}
\hat{\phi} &= \argmin_{\phi} L'(g(X,\phi, \phi) \\
\hat{\psi} &= \argmin_{\psi} R_{\text{emp}}(y, h(X, (\psi, \hat{\phi})))
\end{align*}

We choose to use this notation to illustrate our point that imputation is an integral part of the ERM, not a separate, preliminary process. In particular, it means that in theory its parameters \emph{must} be subjected to cross-validation just like those of the prediction. That is $X_A$ the training data are used to estimate $(\hat{\psi}, \hat{\phi})$ as shown just above. Then, we can compute a prediction $\hat{y}_V = h(X_V, (\hat{\psi}, \hat{\phi}))$ which can be compared to $y_V$ to evaluate the choice of model.

The bottom line is that just like for the prediction, the imputation parameter $\phi$ should be estimated only on the training data and then used on the validation data. As we will see, this raises an issue with the way current imputation methods are implemented.

		\subsection{Unsuitability of current methods}
Over the years, many imputation methods have been proposed, and we describe some in Chapter \ref{imputation}. They have various assumptions and principles but they have one thing in common: they are all implemented (usually in R) via a single function. That function takes a dataset with missing values as an input and returns the dataset completed by the method of choice, \emph{without giving the user any access to the model itself}.

This is a problem because of how cross-validation is supposed to be performed. Normally, one would estimate $(\hat{\psi}_A, \hat{\phi}_A)$ through ERM, and then make a prediction on the validation set as $h(X_V,(\hat{\psi}_{X_A},\hat{\phi}_{X_A}))$. But here, all we have access to is a function $g': X \mapsto g(X, \hat{\phi}_X)$ where $\hat{\phi}_X$ is the optimised parameter for the argument $X$. It is straightforward to see that using such a function, one can no longer separate the estimation of the parameters and the imputation in itself.
	
Incidentally, this issue was mentioned recently \cite{github_sklearn}, as part of a larger effort to port the MICE \cite{MICE_founding} imputation method to Scikit-learn \cite{scikit-learn}. This package is developed for data scientists, who are used to having separate functions for training and prediction. Likewise, several other mentions of this problem arose in the last two years \cite{thread_newdata1}\cite{thread_newdata2}\cite{thread_newdata3} with no mention of it earlier. It seems that as more elaborate imputation methods become popular with the machine learning community, the need arises for a change in the way those methods are implemented. However, it is clear that such changes take time, and for now we can do one of two things: make do with what is available, or build our own implementation to adapt one of these methods to our goals.
	\section{Possible solutions}
		\subsection{Using current implementations}
			\subsubsection{Methods}
There are many ways we can use to approximate the correct procedure, though none of them is theoretically satisfactory. We describe the most natural ones and their caveats.

\paragraph{Grouped imputation} Impute all of the data at once before performing the cross-validation split. That is, estimate $\hat{\phi}_X$ on the whole data then $$\hat{\psi}_{X_A} = \argmin_{\psi} R_{emp}(y, h(X, (\psi, \hat{\phi}_X)))$$
 and $\hat{y}_V = h(X_V, (\psi_{X_A}, \phi_X))$. 

In that case, the parameter $\phi$ is imputed using the validation data. In particular, imputed values in $X_A$ \emph{depend on those in $X_V$}, which is contrary to the basic principles of cross-validation where validation data must be held out during parameter estimation. In theory this could falsify the validation error by making it too optimistic (if some hard-to-predict observations are present in the test set, that would otherwise have high validation error but will not in this case because of our 'cheating').

\paragraph{Separate imputation}
Divide the data first, then impute each dataset separately:
\begin{align*}
\hat{\phi}_{X_A} &= \argmin_{\phi} R_{emp}'(X_A, \phi) \\
\hat{\phi}_{X_V} &= \argmin_{\phi} R_{emp}'(X_V, \phi) \\
\hat{\psi}_{X_A} &= \argmin_{\psi} R_{emp}(y, h(X, (\psi, \hat{\phi}_{X_A}))) \\
\hat{y}_V &= h(X_V, (\psi_{X_A}, \phi_{X_V}))
\end{align*}

Contrarily to grouped imputation, this does not violate cross-validation at all. However, we are using parameter $\phi_{X_V}$ to impute the validation data and then apply a predictive model that uses $\psi_{X_A}$, optimized for data imputed with $\phi_{X_A}$. If sample size is not large enough, it is possible that $\phi_{X_A}$ and $\phi_{X_V}$ will be noticeably different. In that case, there is no guarantee that $\psi_{X_A}$ will still be valid for prediction on the validation dataset that was imputed differently. 

\paragraph{Line by line imputation}
The last option we mention here is to first impute the training data on its own. Then for every line of the validation data we impute it by stacking it with the imputed training data and imputing the whole dataset. Since it is just one line, we can safely assume that the imputation parameters will be those of the training data. The main issue with this method is that is the validation data is rather large, this will be computationally infeasible.

			\subsubsection{Need for a correct implementation}
We want to understand if the alternatives proposed here are good enough to be used if a correct implementation is impossible. To do that, we need to be able to compare these with the correct method. That means that for at least one imputation method we need to build an implementation that allows us to separate the estimation and the imputation. That way we will be able to compare its performance with the other alternatives we propose

Moreover, in addition to this theoretical pursuit, we need this because of what we are trying to achieve with Traumabase: the end goal is to make a recommendation system that can produce a prediction for \emph{a single new patient} arriving to the hospital, ideally without needing to have access to the whole Traumabase data (which is hard to share because it contains sensitive patient information). Without access to the initial training data, this means that only a fully parametric approach can be taken in this particular case (separate imputation is impossible on just one line of data, and the other ones require access to the full data).

Below, we design a very simple imputation method for those purposes.

		\subsection{A new variant: Multivariate Normal Mode with reserved data}
The principle of this imputation is inspired from R package \emph{Amelia} \cite{ref_amelia}, and a large part of the code is from the \emph{norm} package \cite{pkg_norm}. The idea is to assume that both $X_A$ and $X_V$ follow a normal distribution $\mathcal{N}(\mu, \Sigma)$ with unknown parameters.

\paragraph{Parameter estimation}
It is possible to approximate maximum-likelihood estimators for $\mu$ and $\sigma$ iteratively, using the EM algorithm \cite{EM} \cite{em_normal_fit}. First, the missing values of $X$ are imputed randomly to give $X^{(1)}$. Then we repeat the following steps:

For $t=1...$
\begin{enumerate}
\item $(\mu^{(t)}, \Sigma^{(t)})$ are computed as the maximum likelihood estimators on the completed data $X^{(t)}$ for the normal distribution (empirical mean and covariance)
\item $X^{(t+1)}$ is computed by replacing the missing values by their expected value under the new parameters: 
$$X^{(t+1)} = \mathbb{E}(X \vert X^{\text{obs}} ; \mu^{(t)}, \Sigma^{(t)})$$

The conditional expectation is easily derived using the Schur complement \cite{norm_schur}.
\end{enumerate}

Once this converges we have an estimated parameter $\hat{phi} = (\hat{\mu}, \hat{\Sigma})$.

\paragraph{Imputation}
Once we have the parameters, it is very straightforward to get an imputation of the missing data. There are two different ways of doing this:
\begin{itemize}
\item Replace missing values by their expectation conditional on the observed data: 
$$\hat{X} = \mathbb{E}(X \vert X^{\text{obs}} ; \hat{\mu}, \hat{\Sigma})$$
That is, pick the mode of the conditional distribution.
\item Draw the missing values from the conditional distribution
\end{itemize}

In the second case, the \emph{norm} package provides all that is needed. This is what would be used for multiple imputation. However, as we see in Chapter \ref{linreg}, the mode is the optimal choice if we want a single best imputation to use for a prediction model. We implemented this method. We use it below in order to compare the methods mentioned before.

		\subsection{Comparison on simulated data}
In order to compare the methods used for imputation
		
\addtocontents{toc}{\protect\vspace*{\fill}}
\addtocontents{toc}{\protect\newpage} % used to split the toc over two pages		
\addtocontents{toc}{\protect\vspace*{\fill}}

		
\chapter{Error sources and best imputation: the case of linear regression with missing data}
In order to make good decisions for imputation, it is important to understand how it impacts prediction. In order to get an idea on this issue, we solve a very simple case of cross-validated regression with missing data. We found no theoretical results whatsoever about the relationship between imputation and prediction. 
\label{linreg}
	\section{Problem set-up}
We place ourselves in a simple linear regression context with two covariates, where one observation is missing in the training dataset, and one in the validation dataset.

		\subsection{Notations}
			\subsubsection{True data}
The true, unobserved data is a standard regression setup, with the exception that one line of the data is reserved as 'validation data'
\footnote{For this calculation we only need one line of validation data since we will be using expected values.}
, with the rest called 'training data' --- as is standard in predictive contexts (cf Chapter \ref{validation}). The response variable $y$ is a noised linear combination of the covariates in $X$:
\begin{equation*}
\tilde{X}_A = 
\begin{pmatrix}
x_{11} & x_{12} \\
\vdots & \vdots \\
x_{n1} & x_{n2}
\end{pmatrix}
\quad \mathrm{and} \quad
y_A = X_A \beta + \epsilon_A
\quad \mathrm{with} \quad
\epsilon_A \sim \mathcal{N}(0, \sigma^2)
\end{equation*}
\begin{equation*}
\tilde{X_V} = 
\begin{pmatrix}
x_{1}^V & x_{2}^V
\end{pmatrix}
\quad \mathrm{and} \quad
y_V = X_V \beta + \epsilon_V
\quad \mathrm{with} \quad
\epsilon_V \sim \mathcal{N}(0, \sigma^2)
\end{equation*}

The end goal is to learn an estimator on the training set that minimizes the expected loss on the validation set:
$$
L(y_V, \hat{y}_V) = (y_V - \hat{y_V})^2
$$

This is a well known problem with a simple solution. The regression estimator can be expressed as:
$$ \tilde{\beta} = (\tilde{X}_A^T \tilde{X}_A)^{-1} \tilde{X}_A^T y_A $$

with and estimated response
$$ \tilde{y}_V = \tilde{X}_V \tilde{\beta} $$

But there is one caveat: the data is actually not fully observed.

			\subsubsection{Observed data}
What we actually have access to is slightly different: one observation is missing in the training set, as is one of the entries in the testing set. We observe the full $y^A$, but the covariate matrices we actually have access to are:
\begin{equation*}
X^A = 
\begin{pmatrix}
? & x_{12} \\
\vdots & \vdots \\
x_{n1} & x_{n2}
\end{pmatrix}
\quad \mathrm{and} \quad
X^V = 
\begin{pmatrix}
? & x_{2}^V
\end{pmatrix}
\end{equation*}

		\subsection{Imputed data and regression}
			\subsubsection{Principle}
To perform a linear regression similarly to what we did previously, we first have to fill in the blanks: we impute the missing data by replacing them with chosen values $\phi$ and $\psi$ (which we choose using the value of what we observe).
\begin{equation*}
\hat{X}^A = 
\begin{pmatrix}
\phi & x_{12} \\
\vdots & \vdots \\
x_{n1} & x_{n2}
\end{pmatrix}
\quad \mathrm{and} \quad
\hat{X}^V = 
\begin{pmatrix}
\psi & x_{2}^V
\end{pmatrix}
\end{equation*}

This in turn allows us to perform the regression just like before:
$$
\hat{\beta} = (\hat{X}_A^T \hat{X}_A)^{-1} \hat{X}_A^T y_A 
\quad \mathrm{and} \quad
\hat{y}_V (\phi, \psi) = \hat{X}_V \hat{\beta} 
$$

This means that what we really want to minimize is the following risk function:

$$R(\phi, \psi) = \mathbb{E}[(y_V - \hat{y}_V(\phi,\psi))^2 \vert X_A, X_V]$$

where the only thing we choose is our decision rule for $\phi$ and $\psi$.
			\subsubsection{Distribution hypotheses}
Lastly, for this last expression to have any meaning, we need to make some assumption on the distribution of $X$.

We make a very simple hypotheses of a multivariate normal distribution for the covariates --- the parameters are assumed to be known, in real life they would be estimates.
$$
\tilde{X}_A \sim \mathcal{N}(\mu, \Sigma) \quad \tilde{X}_V \sim \mathcal{N}(\mu, \Sigma)
$$

With these, we know the distribution of the missing observations $x_{11}$ and $X_1^V$ conditional of the observed ones and we can try to find the $\phi$ and $\psi$ values that give us the best expected loss.

Now that we have these notations and hypotheses, it is possible to solve the problem of the choice of imputation. Here we give a full expression of the loss and partial resolution.

	\section{Partial resolution}
		\subsection{General loss}
To be able to estimate the expected loss, we break it up into several components
\begin{align*}
L(y_V, \hat{y}_V) &= (y_V - \hat{y_V})^2 &\\
				   &= (\tilde{X}_V \beta + \epsilon_V - \hat{X}_V \hat{\beta})^2 &\\
				   &= (\tilde{X}_V(\beta - \tilde{\beta}) + \tilde{X}_V (\tilde{\beta} - \hat{\beta}) + (\tilde{X}_V - \hat{X}_V) \hat{\beta} + \epsilon_V)^2 & \\
				   &= (\tilde{X}_V (\beta - \tilde{\beta}))^2 & (1) \\
				   & \quad + (\tilde{X}_V (\tilde{\beta}-\hat{\beta}))^2 &(2) \\
				   & \quad + ((\tilde{X}_V - \hat{X}_V) \hat{\beta})^2 &(3) \\
				   & \quad + \tilde{X}_V (\beta - \tilde{\beta}) \tilde{X}_V (\tilde{\beta} - \hat{\beta}) & (4) \\
				   & \quad + \tilde{X}_V (\beta - \tilde{\beta}) (\tilde{X}_V - \hat{X}_V )\hat{\beta} & (5) \\
				   & \quad + \tilde{X}_V (\tilde{\beta} - \hat{\beta}) (\tilde{X}_V - \hat{X}_V) \hat{\beta} & (6)\\
				   & + \epsilon_V^2 &\\
				   & + \epsilon_V K
\end{align*}
Where $K$ is some term that will not matter (because when we take the expectation it will be zero). The risk we want to minimize is the expectation of this loss.
		\subsection{When the validation set is fully observed}
			\subsubsection{Imputation}
The first thing we can easily do is to study the situation where the only missing data is in the validation set. In that case, $\tilde{X}_A = \hat{X}_A$ and so $\tilde{\beta} = \hat{\beta}$, and all we have to choose is $\psi$. In the previously computed loss, it means that terms $(2)$, $(4)$ and $(6)$ are zero.

Furthermore, Cochran's theorem ensures that $(\beta - \tilde{\beta})$ and $\tilde{\beta}$ are independent so term $(5)$ can be factorized and will have zero expectation (since $(\beta - \tilde{\beta})$ has zero expectation).

Term $(1)$ depends only on the true values of $X_V$, independent of $psi$, so the choice of $\psi$  is not impacted by this term.

This leaves us with only term $(3)$, with expectation:
\begin{align*}
\mathbb{E}[((\tilde{X}_V-\hat{X}_V)\tilde{\beta})^2 \vert x^V_2, X_A] &= \mathbb{E}[(x_1^V - \psi)^2 \tilde{\beta}_1^2 \vert x^V_2, X_A] \\
													&= \mathbb{E}[\tilde{\beta}_1^2 \vert x^V_2, X_A] (\mathbb{E}[(x_1^V)^2\vert x^V_2, X_A] - 2\psi \mathbb{E}[x_1^V\vert x^V_2, X_A] + \psi^2)
\end{align*}
Once we are there, we can differentiate this expression to easily derive the optimal expression for $\psi$: $\hat{\psi} = \mathbb{E}[x_1^V\vert x^V_2]$.

Incidentally, this does not use any assumption on the distribution of $X$: this would be true for any joint distribution we choose for the covariates.

			\subsubsection{Expected loss}
First note:
\begin{align*}
\eta = \beta - \tilde{\beta} &= \beta - (\tilde{X}_A^T \tilde{X}_A)^{-1} \tilde{X}_A^T y_A \\
		&= (\tilde{X}_A^T \tilde{X}_A)^{-1} \tilde{X}_A^T (\tilde{X}_A \beta + \epsilon_A) \\
		&= (\tilde{X}_A^T \tilde{X}_A)^{-1} \tilde{X}_A^T \epsilon_A
\end{align*}

That is, the difference between the estimated and real parameter is distributed following some centred normal distribution. Let us denote its covariance matrix by $S = \sigma^2 (\tilde{X}_A^T \tilde{X}_A)^{-1} $.

Now, term $(1)$ can be expressed as :
\begin{align*}
\mathbb{E}[(\tilde{X}_V \eta)^2 \vert x^V_2] &= \mathbb{E}[(x^V_1 \eta_1 + x^V_2 \eta_2)^2\vert x^V_2] \\
											&= S_{11} \mathbb{E}[(x^V_1)^2 \vert x^V_2]  + 2x^V_2 S_{12} \mathbb{E}[x^V_1 \vert x^V_2] + S_{22} (x^V_2)^2
\end{align*}

Term $(3)$ can be expressed as:
\begin{align*}
\mathbb{E}[((\tilde{X}_V - \hat{X}_V) \hat{\beta})^2 \vert x^V_2]&= \mathbb{E}[((x^V_1 - \hat{\psi})\tilde{\beta}_1)^2 \vert x^V_2] \\
									 &= \mathbb{E}[((x^V_1 - \hat{\psi})(\beta_1 + \eta_1))^2 \vert x^V_2] \\
									&= \beta_1^2 \text{Var}[x^V_1 \vert x^V_2] + S_{11}^2 \text{Var}[x^V_1 \vert x^V_2] 
\end{align*}

Finaly, adding the variance of the noise $\epsilon_V$, we see that the expected loss is :
$$
S_{11} \mathbb{E}[(x^V_1)^2 \vert x^V_2]  + 2x^V_2 S_{12} \mathbb{E}[x^V_1 \vert x^V_2] + S_{22} (x^V_2)^2 + 
\beta_1^2 \text{Var}[x^V_1 \vert x^V_2] + S_{11}^2 \text{Var}[x^V_1 \vert x^V_2]  +
\sigma^2
$$

The terms from $(1)$ would be more or less the same if the test data were fully observed. They represent the impact on the prediction of the error made when estimating $\beta$.

 On the other hand, those from $(3)$ are both positive and unique to the incomplete case. They do not depend on the  test data at all. The first term reflects how an error in the imputation of $X_V$ is amplified by the regression coefficient when predicting $y$. The second one shows how errors in the estimation of $\beta$ and of $X_V$ combine when predicting $y$.
 
 Most importantly, the least squares estimator is a strongly consistent one\cite{consistency_linreg}, which implies that the only error terms that matter with large $n$ are those that remain when $\eta$ is set to zero:
 $$ \beta_1^2 \text{Var}[x^V_1 \vert x^V_2] + \sigma^2 $$
 
 The missing data in the validation set adds a term that does not vanish for large $n$.
 
 \paragraph*{When there are more than two covariates}
 An important point is that in a context with more than two covariates, the conditional variance of an observation increases when the number of unobserved variables increases: this means that when the dimension increases, not only do new error terms appear, but the existing ones also increase.
 
 For $X_V$ with $p>2$ covariates, we denote $X_V^{\text{miss}}$ and $X_V^{\text{obs}}$ the missing and observed values in $X_V$. We can now write $(3)$ again:
 \begin{align*}
 \mathbb{E}[((\tilde{X}_V - \hat{X}_V)\beta)^2 \vert X_V^{\text{obs}}] &= 
 		\mathbb{E}[ (\sum \limits_{\substack{i=1 \\i \in X_V^{\text{miss}}}}^{p} (x_i^V - \hat{x}_i^V)\beta_i)^2 \vert X_V^{\text{obs}}] \\
 		&= \mathbb{E}[\sum \limits_{\substack{i=1 \\i \in X_V^{\text{miss}}}}^{p} \sum \limits_{\substack{j=1 \\j \in X_V^{\text{miss}}}}^{p}
 			(x^V_i - \hat{x}^V_i)(x^V_j - \hat{x}^V_j)\beta_i \beta_j \vert X_V^{\text{obs}}]
 \end{align*}
 
 If, as previously, we choose $\hat{x}^V_i = \mathbb{E}[x^V_i \vert X_V^{\text{obs}}]$, we end up with:
 \begin{align*}
 \mathbb{E}[((\tilde{X}_V - \hat{X}_V)\beta)^2 \vert X_V^{\text{obs}}] &= 
 	\sum \limits_{\substack{i,j=1 \\i,j \in X_V^{\text{miss}}}}^{p} \beta_i \beta_j Cov(x^V_i, x^V_j \vert X_V^{\text{obs}})
 \end{align*}
 
		\subsection{When the data is large and the training data is fully observed}
We can suppose that $n$ is large and that we know $\tilde{X_V}$. In that case, we can take the approximation that $\beta = \tilde{\beta}$, that is, the only error in the estimation of $\beta$ comes from the missing data in the training set. Then, the only error term that is nonzero is $(2)$:
$$ (\tilde{X}_V (\beta - \tilde{\beta}))^2$$
\todo{To complete}
	\section{Consequences}
		\subsection{Implications}
\paragraph{Theoretical consequences}
If the results above are any indication as to how things go in more complex settings, there are some interesting implications. 

First, this confirms our intuition that using the conditional mode is the right thing to do to impute the validation data. This is important, especially as most imputation packages were made for multiple imputation and draw from the conditional distribution instead.

Second, there is a major asymmetry between the training and validation dataset: in the training set, every line works together with the others to help estimate some parameter. In particular, this means that even if all of the imputations are imperfect, with enough observation we will obtain a very satisfactory estimate of the parameters (this is similar to the case of statistical inference with missing data, where under some assumptions the estimators are asymptotically consistent \cite{rubin_ignorability}). More data adds information, and even if it is incomplete it helps with the estimation.

The validation data is in a completely different situation. Missing data in the validation dataset adds error terms to the data that can be very large and do not vanish asymptotically. Intuitively, even if we have exactly the right $\beta$ for regression, any error in the estimation of the data will be directly reflected as a prediction error proportional to the regression coefficient, while in the training set this effect is much more indirect. The expected loss will be the same for every line on the validation set, so adding more validation lines will do nothing to reduce the mean error if they also have missing data. This is important not only from the standpoint of model selection, but also because the validation error gives us an idea of how our model will perform on real-world data: missing data in the data we use for prediction can be a much more severe issue than missing data in our training database.

\paragraph{Implications for our data}

This could actually be good news. Take the example of Traumabase and haemorrhagic shock prediction: part of what our results mean is that the Traumabase can indeed be used to build a prediction tool, even if it has a significant amount of missing data. If the missing data is mostly due to errors of recording, this may mean that it is available in the real world when a doctor uses the tool: if the data used to make important predictions (that is, not a posteriori from the base but in a hospital when a patient needs care) can be kept full, then our estimates will have a chance of being very good, as long as we built our model with a large enough database.

The flip size, of course, is that when data is indeed missing from the validation dataset, there is little we can do to offset the resulting penalty. The missing value has a natural variability, even when controlling for every other observed variable, so even our best guess could have a high error. 

\subsection{Partial multiple imputation}
This suggests an idea that could help mitigate this drawback while keeping the computational cost rather low. Using $X_A$ the training data, we can estimate parameters $\phi, \psi, \beta$ as usual. Then, keeping these parameters constant, we can make draws from the conditional distribution of $X_V$ and make predictions on the datasets generated this way. Using the quantiles of these predictions, we can build intervals that approximate the possible location of the true value of $y$ and account for uncertainty. The idea behind this is that it is not necessary to multiply impute $X_A$ because with lare $n$ the estimated parameters will not really vary. It is also computationally intensive because it means we have to fit a new model for every imputed dataset. With our method, just one fit is needed and we only perform multiple predictions, which are usually cheaper.

To illustrate these results, we perform an analysis on some very simple simulated data, and check if the properties we derived are visible.
		\subsection{Verification with simulated data}

		
\chapter{Analysis: imputing the Traumabase data for prediction}
\label{analysis}
	\section{Criteria for evaluation}
	\section{Choosing the imputation method}
		Mention the fact that most methods have the same performance (even mean)
		
		Imputation done before model selection (cf congeniality)
	\section{Multiple imputation}
	
\chapter{Results}
\label{results}

\chapter*{Conclusion}
\addcontentsline{toc}{chapter}{Conclusion}

\bibliography{medical,stats,methods}
\end{document}