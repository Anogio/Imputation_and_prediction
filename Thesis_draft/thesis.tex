\documentclass[12pt, a4paper]{memoir}
\usepackage{import}

\usepackage{preamble}

\begin{document}
\pagenumbering{roman}

\AtBeginShipoutNext{\AtBeginShipoutNext{\AtBeginShipoutDiscard}}
\begin{titlingpage}
	\newcommand{\HRule}{\rule{\linewidth}{0.5mm}} % Defines a new command for the horizontal lines, change thickness here
	
	\center % Center everything on the page
	 
	%----------------------------------------------------------------------------------------
	%	HEADING SECTIONS
	%----------------------------------------------------------------------------------------
	
	\textsc{\LARGE University of Oxford}\\[1.5cm] % Name of your university/college
	\textsc{\Large MSc in Statistical Science}\\[0.5cm] % Major heading such as course name
	\textsc{\large Final thesis}\\[0.5cm] % Minor heading such as course title
	
	%----------------------------------------------------------------------------------------
	%	TITLE SECTION
	%----------------------------------------------------------------------------------------
	
	\HRule \\[0.4cm]
	{ \huge \bfseries Missing data imputation for Haemorrhagic shock prediction}\\[0.4cm] % Title of your document
	\HRule \\[1.5cm]
	 
	%----------------------------------------------------------------------------------------
	%	AUTHOR SECTION
	%----------------------------------------------------------------------------------------
	
	\begin{minipage}{0.4\textwidth}
	\begin{flushleft} \large
	\emph{Author:}\\
	Antoine \textsc{Ogier} % Your name
	\end{flushleft}
	\end{minipage}
	~
	\begin{minipage}{0.4\textwidth}
	\begin{flushright} \large
	\emph{Supervisor:} \\
	Pr. Julie \textsc{Josse} \\(École polytechnique) \\% Supervisor's Name
	Pr. Geoff \textsc{Nicholls} \\(University of Oxford) % Supervisor's Name
	
	\end{flushright}
	\end{minipage}\\[1cm]
	
	% If you don't want a supervisor, uncomment the two lines below and remove the section above
	%\Large \emph{Author:}\\
	%John \textsc{Smith}\\[3cm] % Your name
	
	%----------------------------------------------------------------------------------------
	%	DATE SECTION
	%----------------------------------------------------------------------------------------
	
	{\large September 2018}\\[1cm] % Date, change the \today to a set date if you want to be precise
	
	%----------------------------------------------------------------------------------------
	%	LOGO SECTION
	%----------------------------------------------------------------------------------------
	
	\includegraphics[scale=1.2]{Resources/logo}\\[1cm] % Include a department/university logo - this will require the graphicx package
	%----------------------------------------------------------------------------------------
	%\vfill % Fill the rest of the page with whitespace
\end{titlingpage}

\begin{vplace}[0.7]
\begin{abstract}
Hemorrhagic shock is a condition that can be life-threatening but that has much higher survival rates if treated early. It is also quite difficult to detect. Because of this, there is a strong case for a tool that predicts it based on prehospital measurement made on trauma patients. 

The Traumabase dataset provides a large history of such measurements, and could be used to learn a model to predict hemorrhagic shock (Chapter \ref{data}). However, the presence of missing data complicates the task. In this work, we specifically explore one way to handle missing data: imputation of unobserved values (Chapter \ref{imputation}). In a context where the final goal is prediction on new real-world patients, rather than parameter estimation, there are some important differences that we investigate. In particular, current implementations of imputation methods need to be modified to work in such case (Chapter \ref{validation}). The possible presence of missing values for new patients at the time of prediction (in addition to those in the records) means that some issues beyond parameter estimation appear (Chapters \ref{linreg} and \ref{empirical}).

After investigating the issues linked to imputation in this context, we go back to the Traumabase and estimate its potential for hemorrhagic shock prediction (Chapter \ref{analysis}).

\end{abstract}
\end{vplace}

\newpage
\vspace*{\fill}
{\centering\huge\bfseries Acknowledgements\par}
\bigskip
\noindent I would first like to thank Pr.\ Julie Josse and Pr.\ Geoff Nicholls who supervised my work and were always available to talk about the challenges I encountered.

I would also like to thank Pr.\ Jean-Pierre Nadal, Dr.\ Sophie Hamada and Dr.\ Tobias Gauss for their expertise on the Traumabase and their friendliness throughout the project.

Thank you to Morgane for proofreading my work.
\vspace*{\fill}
\newpage

%\vspace*{\fill}
\tableofcontents*
\vspace*{\fill}

\addtocontents{toc}{\protect\vspace*{\fill}}
%\chapter*{Introduction}
\pagenumbering{arabic}
%\addcontentsline{toc}{chapter}{Introduction}

\chapter{Goal and data}
\label{data}
 \import{.}{1_chapter1}

\chapter{Imputation}
\label{imputation}
\import{.}{2_chapter2}
		
\chapter{Methodology: imputation and the validation split}
\label{validation}
\import{.}{3_chapter3}
		
\addtocontents{toc}{\protect\vspace*{\fill}}
\addtocontents{toc}{\protect\newpage} % used to split the toc over two pages		
\addtocontents{toc}{\protect\vspace*{\fill}}		
		
\chapter{Impact of missing data: the case of linear regression}
\label{linreg}
\import{.}{4_chapter4} 

%\chapter{Imputation and prediction: Empirical findings}
%\label{empirical}
%\import{.}{5_chapter5}

\chapter{Imputing the Traumabase data for prediction}
\label{analysis}
\import{.}{6_chapter6}

\chapter*{Conclusion}
\addcontentsline{toc}{chapter}{Conclusion}
In this work, we investigated the possibility to predict hemorrhagic shock in trauma patients when imputing the missing values, both in the database and at the time of prediction. We tried to understand how this taks should be performed, and found that to use existing imputation methods in the real world, they must be modified so that data for a single new patient can be imputed (as opposed to a block of many lines). We implemented such a method and compared it to other alternatives that can be used for model selection and to not require a new implementation, and found that grouped imputation seemed to be a good indicator of a methods true performance.

We also studied a simplified case to understant the specificities of our task. It is clear that the missing values in the validation data/new patients means that estimating a good regression parameter on the training data is not our only task (we also need to choose how to impute the new data). In favorable cases (with MCAR data), we show that these two objectives go hand in hand: when the imputation is done using the true conditional expectation and the estimation uses a consistent full-data method on the imputed data, then the imputation error is minimal and the parameter estimation remains consistent on the imputed data. 

Additionally, if we have some control over the amount of missing data on the training data or the real world data, but not both (which can happen, as the causes of missingness can be separate: a measurement can be made by the doctor and be available at the time of diagnostic, then not be recorded into the base. On the contrary, a measurement can be delayed because of lack of time or instruments and be completed once the patient reaches the hospital), then one should focus on limiting misingness in the real world data because missing information at the time of prediction reflects much more directly in the validation error. On the contrary, decreasing only the amount of missing training data may actually have a negative effect when it comes to predicting new data with missing values.

The final evaluation of imputation on the Traumabase data gave us mixed results. Indeed, it appears that the joint SAEM regression gives significantly better results than prediction after imputation. Still, performing imputation allowed us to compare many prediction methods with minimal efforts, which had the potential to guide us in our future choices of imputation.

In the future, it would be worthwile to quantify formally the gap in predictive performance between joint optimization, and imputation followed by prediction with the same hypotheses (for instance, SAEM has the same distribution hypotheses as normal imputation followed by logistic regression: it is only the choice of parameters that differs). In the investigation of HS prediction, an important next step would be to understand why all imputation methods seem to perform similarly.
\begin{appendices}
\chapter{Simulated normal data}
\label{simulation}
\import{.}{Appendix1}

\chapter{Abalone data}
\label{abalone}
\import{.}{Appendix2}
\end{appendices}

\bibliography{medical,stats,methods}
\end{document}