\documentclass[12pt, a4paper]{memoir}
\usepackage{import}

\usepackage{preamble}

\begin{document}
\pagenumbering{roman}

\AtBeginShipoutNext{\AtBeginShipoutNext{\AtBeginShipoutDiscard}}
\begin{titlingpage}
	\newcommand{\HRule}{\rule{\linewidth}{0.5mm}} % Defines a new command for the horizontal lines, change thickness here
	
	\center % Center everything on the page
	 
	%----------------------------------------------------------------------------------------
	%	HEADING SECTIONS
	%----------------------------------------------------------------------------------------
	
	\textsc{\LARGE University of Oxford}\\[1.5cm] % Name of your university/college
	\textsc{\Large MSc in Statistical Science}\\[0.5cm] % Major heading such as course name
	\textsc{\large Final thesis}\\[0.5cm] % Minor heading such as course title
	
	%----------------------------------------------------------------------------------------
	%	TITLE SECTION
	%----------------------------------------------------------------------------------------
	
	\HRule \\[0.4cm]
	{ \huge \bfseries Missing data imputation for Haemorrhagic shock prediction}\\[0.4cm] % Title of your document
	\HRule \\[1.5cm]
	 
	%----------------------------------------------------------------------------------------
	%	AUTHOR SECTION
	%----------------------------------------------------------------------------------------
	
	\begin{minipage}{0.4\textwidth}
	\begin{flushleft} \large
	\emph{Author:}\\
	Antoine \textsc{Ogier} % Your name
	\end{flushleft}
	\end{minipage}
	~
	\begin{minipage}{0.4\textwidth}
	\begin{flushright} \large
	\emph{Supervisor:} \\
	Pr. Julie \textsc{Josse} \\(École polytechnique) \\% Supervisor's Name
	Pr. Geoff \textsc{Nicholls} \\(University of Oxford) % Supervisor's Name
	
	\end{flushright}
	\end{minipage}\\[1cm]
	
	% If you don't want a supervisor, uncomment the two lines below and remove the section above
	%\Large \emph{Author:}\\
	%John \textsc{Smith}\\[3cm] % Your name
	
	%----------------------------------------------------------------------------------------
	%	DATE SECTION
	%----------------------------------------------------------------------------------------
	
	{\large September 2018}\\[1cm] % Date, change the \today to a set date if you want to be precise
	
	%----------------------------------------------------------------------------------------
	%	LOGO SECTION
	%----------------------------------------------------------------------------------------
	
	\includegraphics[scale=1.2]{Resources/logo}\\[1cm] % Include a department/university logo - this will require the graphicx package
	%----------------------------------------------------------------------------------------
	%\vfill % Fill the rest of the page with whitespace
\end{titlingpage}

\begin{vplace}[0.7]
\begin{abstract}
Hemorrhagic shock is a condition that can be life-threatening but that has much higher survival rates if treated early. It is also quite difficult to detect. Because of this, there is a strong case for a tool that predicts it based on prehospital measurement made on trauma patients. 

The Traumabase dataset provides a large history of such measurements, and could be used to learn a model to predict hemorrhagic shock (Chapter \ref{data}). However, the presence of missing data complicates the task. In this work, we specifically explore one way to handle missing data: imputation of unobserved values (Chapter \ref{imputation}). In a context where the final goal is prediction on new real-world patients, rather than parameter estimation, there are some important differences that we investigate. In particular, current implementations of imputation methods need to be modified to work in such case (Chapter \ref{validation}). The possible presence of missing values for new patients at the time of prediction (in addition to those in the records) means that some issues beyond parameter estimation appear (Chapters \ref{linreg} and \ref{empirical}).

After investigating the issues linked to imputation in this context, we go back to the Traumabase and estimate its potential for hemorrhagic shock prediction (Chapter \ref{analysis}).

\end{abstract}
\end{vplace}

\newpage
\vspace*{\fill}
{\centering\huge\bfseries Acknowledgements\par}
\bigskip
\noindent Lorem ipsum dolot sit amet
\vspace*{\fill}
\newpage

%\vspace*{\fill}
\tableofcontents*
\vspace*{\fill}

\addtocontents{toc}{\protect\vspace*{\fill}}
%\chapter*{Introduction}
\pagenumbering{arabic}
%\addcontentsline{toc}{chapter}{Introduction}

\chapter{Goal and data}
\label{data}
 \import{.}{1_chapter1}

\chapter{Imputation}
\label{imputation}
\import{.}{2_chapter2}
		
\chapter{Methodology: imputation and the validation split}
\label{validation}
\import{.}{3_chapter3}
		
\addtocontents{toc}{\protect\vspace*{\fill}}
\addtocontents{toc}{\protect\newpage} % used to split the toc over two pages		
\addtocontents{toc}{\protect\vspace*{\fill}}		
		
\chapter{Impact of missing data: the case of linear regression}
\label{linreg}
\import{.}{4_chapter4} 

\chapter{Imputation and prediction: Empirical findings}
\label{empirical}
\import{.}{5_chapter5}

\chapter{Analysis: imputing the Traumabase data for prediction}
\label{analysis}
\import{.}{6_chapter6}

\chapter*{Conclusion}

\addcontentsline{toc}{chapter}{Conclusion}

\begin{appendices}
\chapter{Simulated normal data}
\label{simulation}
\import{.}{Appendix1}

\chapter{Abalone data}
\label{abalone}
\import{.}{Appendix2}
\end{appendices}

\bibliography{medical,stats,methods}
\end{document}