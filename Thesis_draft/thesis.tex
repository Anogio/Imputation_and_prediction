\documentclass[12pt, a4paper]{memoir}
\usepackage{import}

\usepackage{preamble}

\begin{document}
\pagenumbering{roman}

\AtBeginShipoutNext{\AtBeginShipoutNext{\AtBeginShipoutDiscard}}
\begin{titlingpage}
	\newcommand{\HRule}{\rule{\linewidth}{0.5mm}} % Defines a new command for the horizontal lines, change thickness here
	
	\center % Center everything on the page
	 
	%----------------------------------------------------------------------------------------
	%	HEADING SECTIONS
	%----------------------------------------------------------------------------------------
	
	\textsc{\LARGE University of Oxford}\\[1.5cm] % Name of your university/college
	\textsc{\Large MSc in Statistical Science}\\[0.5cm] % Major heading such as course name
	\textsc{\large Final thesis}\\[0.5cm] % Minor heading such as course title
	
	%----------------------------------------------------------------------------------------
	%	TITLE SECTION
	%----------------------------------------------------------------------------------------
	
	\HRule \\[0.4cm]
	{ \huge \bfseries Missing data imputation for Haemorrhagic shock prediction}\\[0.4cm] % Title of your document
	\HRule \\[1.5cm]
	 
	%----------------------------------------------------------------------------------------
	%	AUTHOR SECTION
	%----------------------------------------------------------------------------------------
	
	\begin{minipage}{0.4\textwidth}
	\begin{flushleft} \large
	\emph{Author:}\\
	Antoine \textsc{Ogier} % Your name
	\end{flushleft}
	\end{minipage}
	~
	\begin{minipage}{0.4\textwidth}
	\begin{flushright} \large
	\emph{Supervisor:} \\
	Pr. Julie \textsc{Josse} \\(École polytechnique) \\% Supervisor's Name
	Pr. Geoff \textsc{Nicholls} \\(University of Oxford) % Supervisor's Name
	
	\end{flushright}
	\end{minipage}\\[1cm]
	
	% If you don't want a supervisor, uncomment the two lines below and remove the section above
	%\Large \emph{Author:}\\
	%John \textsc{Smith}\\[3cm] % Your name
	
	%----------------------------------------------------------------------------------------
	%	DATE SECTION
	%----------------------------------------------------------------------------------------
	
	{\large September 2018}\\[1cm] % Date, change the \today to a set date if you want to be precise
	
	%----------------------------------------------------------------------------------------
	%	LOGO SECTION
	%----------------------------------------------------------------------------------------
	
	\includegraphics[scale=1.2]{Resources/logo}\\[1cm] % Include a department/university logo - this will require the graphicx package
	%----------------------------------------------------------------------------------------
	%\vfill % Fill the rest of the page with whitespace
\end{titlingpage}

\begin{vplace}[0.7]
\begin{abstract}
Lorem ipsum dolot sit amet nunc cui Brexit.
\end{abstract}
\end{vplace}

\newpage
\vspace*{\fill}
{\centering\huge\bfseries Acknowledgements\par}
\bigskip
\noindent Lorem ipsum dolot sit amet
\vspace*{\fill}
\newpage

%\vspace*{\fill}
\tableofcontents*
\vspace*{\fill}

\addtocontents{toc}{\protect\vspace*{\fill}}
\chapter*{Introduction}
Main outline: Haemorrhagic shock is a condition that can be life-threatening but that has much higher survival rates if treated early. In addition, doctors tend to have a fairly bad record of detecting it. Because of this, we want to build a tool that predicts it based on measurement on trauma patients. For this we use Traumabase, a large patient-records database. Problem: lots of missing data in it. Solutions: use algorithms specifically made for missing data or impute missing data. Nice thing about imputation: once it is done, you can use any existing method -> here we only work on imputation (not so much the prediction part).

We present the data (Chapter \ref{data}), then present the state of the art in imputation (Chapter \ref{imputation}). Then, we derive some theory on imputation when it is performed with prediction as a goal (Chapters \ref{validation} and \ref{linreg}). We then come back to the data to apply what we learn, in order to choose the best imputation method for this problem (Chapter \ref{analysis}) and present our final results (Chapter \ref{results}).
\pagenumbering{arabic}
\addcontentsline{toc}{chapter}{Introduction}

\chapter{Goal and data}
\label{data}
 \import{.}{1_chapter1}

\chapter{Imputation}
\label{imputation}
\import{.}{2_chapter2}
		
\chapter{Methodology: imputation and the validation split}
\label{validation}
\import{.}{3_chapter3}
		
\addtocontents{toc}{\protect\vspace*{\fill}}
\addtocontents{toc}{\protect\newpage} % used to split the toc over two pages		
\addtocontents{toc}{\protect\vspace*{\fill}}		
		
\chapter{Impact of missing data: the case of linear regression}
\label{linreg}
\import{.}{4_chapter4} 

\chapter{Imputation and prediction: Empirical results}
\label{empirical}
\import{.}{5_chapter5}

\chapter{Analysis: imputing the Traumabase data for prediction}
\label{analysis}
	\section{Criteria for evaluation}
	\section{Choosing the imputation method}
		Mention the fact that most methods have the same performance (even mean)
		
		Imputation done before model selection (cf congeniality)
	\section{Methodology}
	
	\section{Results}
	\label{results}

\chapter*{Conclusion}
\addcontentsline{toc}{chapter}{Conclusion}

\bibliography{medical,stats,methods}
\end{document}